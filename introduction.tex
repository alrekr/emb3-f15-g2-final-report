\section{Introduction}
\subsection{Project background and scope}
In many applications an autonomous UAV which can follow a preprogrammed route and return to a base to recharge, is sought after. One case could be a farmer. As the legislation is now the farmer must inspect his animals daily. This prevents the use of some remote areas which is otherwise perfect for the farmer’s use, such as small islands or remote field with good grass. Areas that take time, money or effort to get to in order to inspect the animals.

Our vision is to develop a system which allows a UAV to take off, fly a preprogrammed route if conditions allow flying, return to the base using GNSS and then be guided by computer vision to land and recharge on a ground base.

\subsection{Problem formulation}
The problems in this project will be to:

\begin{itemize}
	\item Modify Pixhawk firmware to be controlled automatically from an external computer;
	\item Build/design/construct a landing-platform with mechanical autofit;
	\item Design a tracking-system that outputs 3D-positions; and,
	\item Design a program that analyze 3D-positions and outputs control-signals.
\end{itemize}

\subsection{Related work}
We found a project done by Weiwei Kong, Daibing Zhang, Xun Wang, Zhiwen Xian and Jianwei Zhang, researchers at National University of Defence Technology in Changsha, China and University of Hamburg. This study used a stereo infrared setup placed on the ground which is then used to guide a UAV (multirotor or fixed wing) to land in a designated area. The idea is the same as ours, though for our use there is no need for such advanced equipment. 

The project from NUDT has a great deal of flexibility and can support a multitude of UAVs. This project is based on a single UAV, and it needs a high degree of precision. The general idea is the same, but there are differences. The article can be found at \url{http://ieeexplore.ieee.org/xpls/abs_all.jsp?arnumber=6696776}

\subsection{Aim}
The group will focus on the ground base, landing system and guidance of the drone. We decided to use an IRIS UAV which is controlled by a Pixhawk and then modify the landing gear to fit our concept. We will not develop a charging system as there are already plenty of these, but we will show connectivity and thereby simulate charging.
 
The GNSS is not expected to be precise enough for landing on the platform. It will mainly be used to guide the UAV back in the region of the landing platform. In this region a camera in the landing platform will start tracking the position of the UAV. The landing platform will then take control of the UAV (by telemetry), and start landing the drone based on the vision tracking.
\begin{figure}
	\centering
	\includegraphics[width=0.8\textwidth]{imgs/introduction}
	\caption{Basic idea of how the system could be build}
\end{figure}